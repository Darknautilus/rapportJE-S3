\chapter{Présentation du jeu}

Dans le cadre de notre seconde année passée au département Informatique de l'IUT de Blagnac, nous avons eu l'occasion de mettre à profit nos connaissances acquises en terme de gestion d'entreprise. Durant deux jours et demi, nous avons participé à un « jeu d'entreprise », une simulation qui nous a permis de mesurer nos capacités décisionnelles en concurrence avec trois autres équipes.

Nous avons pu voir un large pan de la direction d'une entreprise : gestion de la production, des ressources humaines, ou encore de la communication. Les clés que nous possédions nous ont permis d'étudier le marché composé de nos quatre entreprises, et de prendre en conséquence des décisions adaptées, le but étant de redresser au mieux l'entreprise que l'on nous avait laissé. Ainsi, il a été nécessaire d'élaborer une stratégie fondée sur des prévisions de comportement du marché, tout en étant réactifs aux changements imprévus qui pouvaient survenir.

Pour réaliser notre objectif, nous avons dû mobiliser notre capacité à travailler en équipe. Au fur et à mesure de la progression du jeu, les qualités de chacun se sont affirmées, ce qui nous a donné des indications sur la répartition des rôles, afin de travailler plus efficacement. En effet, la complémentarité est pour nous essentielle, et nous avons la conviction que chacun peut apporter au groupe, à condition qu'il trouve sa place.